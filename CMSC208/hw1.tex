\documentclass{article}
\usepackage{amsthm}
\usepackage[utf8]{inputenc}
\begin{document}
Nathan Helms: CMSC208: HW1
\begin{enumerate}
\item
$$\sum_{i=1}^{n} i^3 = \frac{1}{4}n^2\left(n+1\right)^2$$
\newline
Base Case:
$$\sum_{i=1}^{1} i^3 = \frac{1}{4}1^2\left(1+1\right)^2 = \frac{1}{4}*2^2 = \frac{4}{4} = 1$$
Assume:
$$\sum_{i=1}^{k} i^3 = \frac{1}{4}k^2\left(k+1\right)^2$$
What we Want:
$$\frac{1}{4}\left(k+1\right)^2\left(k+2\right)^2$$
Inductive Step:
$$\sum_{i=1}^{k+1} i^3 = \left(\sum_{i=1}^{k} i^3\right)+\left(k+1\right)^3$$
$$\sum_{i=1}^{k+1} i^3 = \frac{1}{4}k^2\left(k+1\right)^2+\left(k+1\right)^3$$
$$\sum_{i=1}^{k+1} i^3 = \frac{4\left(k+1\right)^3}{4}+\frac{k^2\left(k+1\right)^2}{4}$$
$$\sum_{i=1}^{k+1} i^3 = \frac{\left(k+1\right)^2\left(4\left(k+1\right)+k^2\right)}{4}$$
$$\sum_{i=1}^{k+1} i^3 = \frac{\left(k+1\right)^2\left(4k+4+k^2\right)}{4}$$
$$\sum_{i=1}^{k+1} i^3 = \frac{\left(k+1\right)^2\left(k+2\right)^2}{4}$$
$$\sum_{i=1}^{k+1} i^3 = \frac{\left(k+1\right)^2\left(k+1+1\right)^2}{4}$$
Which is what we wanted to show
\item
a) The sum of any three consecutive integers is even\newline
Disproven by Counterexample:\newline
$$4+5+6 = 15$$
which is an odd sum.\newline
b) The product of any three consecutive integers is even\newline
Proven by Construction:\newline
$$4*5*6 = 120$$
$$1*2*3 = 6$$
$$2*3*4 = 24$$
which are all even products.\newline
\item
Applying the Pigeonhole principle to this problem leaves us with:
$$7/2 = 3.5$$
\item
Compute the following
$$\sum_{i=1}^{10} \left(-1\right)^{i}i = 5$$
$$\sum_{i=1}^{1000} \left(-1\right)^{i}i = 500$$
\item
Explain the symbolic expression:\newline
For all positive integers, n, in set E, there exists two prime numbers, p1 and p2, of set P, that add together to equal n.
\item
Proven by Construction:\newline
if x = 0, and n = 1
$$\left(1+0\right)^1\geq1+\left(1\right)\left(0\right) = 1 \geq 1 = TRUE$$
if x = 100, and n = 5
$$\left(1+100\right)^5\geq1+\left(5\right)\left(100\right) = 10,510,100,501 \geq 500 = TRUE$$
\end{enumerate}
\end{document}
